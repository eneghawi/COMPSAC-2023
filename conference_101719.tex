\documentclass[conference]{IEEEtran}
\IEEEoverridecommandlockouts
% The preceding line is only needed to identify funding in the first footnote. If that is unneeded, please comment it out.
\usepackage{cite}
\usepackage{amsmath,amssymb,amsfonts}
\usepackage{algorithmic}
\usepackage{graphicx}
\usepackage{textcomp}
\usepackage{xcolor}
\usepackage{url}
\usepackage{soul}
\usepackage{multirow}
\usepackage{lmodern}
\usepackage{framed}
\usepackage{color}
\usepackage{booktabs}
\usepackage{anyfontsize}

\def\BibTeX{{\rm B\kern-.05em{\sc i\kern-.025em b}\kern-.08em
    T\kern-.1667em\lower.7ex\hbox{E}\kern-.125emX}}
\begin{document}
\nocite{*}

\title{Conference Paper Title*\\
% {\footnotesize \textsuperscript{*}Note: Sub-titles are not captured in Xplore and
% should not be used}
% \thanks{Identify applicable funding agency here. If none, delete this.}
}

\author{
\IEEEauthorblockN{Elie Neghawi}
\IEEEauthorblockA{\textit{Electrical and Computer Engineering} \\
\textit{Concordia University}\\
Montreal, Canada \\
e\_negh@encs.concordia.ca}

\and

\IEEEauthorblockN{Yan Liu}
\IEEEauthorblockA{\textit{Electrical and Computer Engineering} \\
\textit{Concordia University}\\
Montreal, Canada \\
yan.liu@concordia.ca}
}

\maketitle

\begin{abstract}
	Adoption of AI systems has been widely used across multiple industry domains at an alerting rate without the focus on it's ethical concerns. To address those concerns, there are an increase number of AI ethics frameworks that have been suggested recently that focus on the algorithmic level rather on the systems level. Nonetheless, some of the system level approaches developed mostly cover a single level governance pattern of the system components in the entire software design lifecycle. However, the need to go beyond the single level  system design AI ethics frameworks to allow not only a better responsible-AI-by-design, but also a trustworthy process patterns that abstract and link the underlying layers of responsible AI on each and every level. This paper illustrates a principal-to-practice guide of the multi-level governance within organizations across the globe for AI ethics frameworks. We outline three main areas of gap in organizations for AI ethics frameworks. Consecutively, we propose a multi-level governance pattern for responsible AI systems within organizations which is participatory, iterative, flexible and operationalizable that target those three main gap areas. Finally, to assist practitioners to apply the multi-level governance AI in organizations and the impact that it has on the industry level, we will translate into effective and responsible AI practices.
\end{abstract}

\begin{IEEEkeywords}
AI, AI ethics, trustworthy AI, AIMLOps, AIOps, software engineering, software architecture, pattern, best practice
\end{IEEEkeywords}

\section{Introduction}
Aritificial Intelligence (AI) reshaped our lives, helped people make better predictions and take more informed and wise decisions. However, these high tech are still in there infancy, and there remains much promise for AI to promote innovation and address global challenges that people face.

Consecutively, ethical concerns and anxieties are fuelling around AI \cite{DBLP}. There are lots of enquiries on the trustworthiness and adoption of AI systems, including concerns about exacerbating inequality, digital divide, climate change and market concentration. Additionally, there are concerns that the use of AI may compromise human rights and values such as privacy. To address these concerns and ensure the responsible development and use of AI, a collaborative effort involving multiple stakeholders and international cooperation issued guidelines and ethical principles. Despite the creation of ethical guidelines for AI development inside organization, it can be challenging for developers to apply these principles in practical situations. These principles are often abstract and may not provide clear direction for specific implementation \cite{abs-2111-09478}. Therefore, more specific and actionable guidelines are needed to assist developers in implementing ethical considerations in their AI systems. It is important to bridge the gap between ethical principles and the algorithms used in AI systems to ensure responsible development. However, The architecture of an AI ecosystem consists of three layers: AI software supply chain, AI system, and operation infrastructure. It is challenging to show the contribution of each.

One work that was proposed is Responsible AI Pattern Catalogue \cite{catalogue}, which takes a pattern-oriented approach to promoting responsible AI in practice. Instead of solely focusing on ethical principles or AI algorithms, this catalogue focuses on design patterns that practitioners can apply to ensure that their AI systems are responsible throughout the software development process. The catalogue is organized into three categories: 1) governance patterns to establish multi-level governance, 2) process patterns to establish trustworthy development processes, and 3) product patterns to integrate responsible design into AI systems. In addition, it focuses on all aspect of the ecosystem (Industry-level, Organization-level and Team-level) without the planning of the design and the developement tools to support the navigation and utilisation of the Responsible AI pattern catalogue.

In this paper, we take a different approach by focusing on the organization-level patterns at the system level rather than just the ethical principles or AI algorithms. This approach aims to integrate responsible design in organizations into final AI products by looking at the bigger picture and the design patterns that shape the system as a whole. This is done with the intention of bridging the gap between the organizational-level and team-level and facilitating navigation. We start off by looking at the main three levels of an organization with the addition to the team-level and examine the current available methods \cite{Shneiderman, ShneidermanRespo, Jana, Hussain, roadmap}. Then we make the links on where those methods meets and create the best practices using the multi-level governance patterns at the organization level.  Our main research questions is as follows:

\vskip 0.1in
\vskip 0.1in

\noindent\fbox{%
    \parbox{.48\textwidth}{%
     What is the multi-level governance pattern proposed for responsible AI systems within organizations?}%
}

\vskip 0.1in
\vskip 0.1in

The main contributions of this paper are as follows:
\begin{itemize}
\item Find the link between Team-level governance patterns with the Organization-level patterns.
\item Suggest navigation and utilisation Team-level governance patterns with the Organization-level patterns.
\item Explore a case study that suits this type of multi-level governance pattern.
\end{itemize}
\smallskip

\section{Methodology}

\subsection{Maintaining the Integrity of the Specifications}

The IEEEtran class file is used to format your paper and style the text. All margins, 
column widths, line spaces, and text fonts are prescribed; please do not 
alter them. You may note peculiarities. For example, the head margin
measures proportionately more than is customary. This measurement 
and others are deliberate, using specifications that anticipate your paper 
as one part of the entire proceedings, and not as an independent document. 
Please do not revise any of the current designations.

\section{Prepare Your Paper Before Styling}
Before you begin to format your paper, first write and save the content as a 
separate text file. Complete all content and organizational editing before 
formatting. Please note sections \ref{AA}--\ref{SCM} below for more information on 
proofreading, spelling and grammar.

Keep your text and graphic files separate until after the text has been 
formatted and styled. Do not number text heads---{\LaTeX} will do that 
for you.

\subsection{Abbreviations and Acronyms}\label{AA}
Define abbreviations and acronyms the first time they are used in the text, 
even after they have been defined in the abstract. Abbreviations such as 
IEEE, SI, MKS, CGS, ac, dc, and rms do not have to be defined. Do not use 
abbreviations in the title or heads unless they are unavoidable.

\subsection{Units}
\begin{itemize}
\item Use either SI (MKS) or CGS as primary units. (SI units are encouraged.) English units may be used as secondary units (in parentheses). An exception would be the use of English units as identifiers in trade, such as ``3.5-inch disk drive''.
\item Avoid combining SI and CGS units, such as current in amperes and magnetic field in oersteds. This often leads to confusion because equations do not balance dimensionally. If you must use mixed units, clearly state the units for each quantity that you use in an equation.
\item Do not mix complete spellings and abbreviations of units: ``Wb/m\textsuperscript{2}'' or ``webers per square meter'', not ``webers/m\textsuperscript{2}''. Spell out units when they appear in text: ``. . . a few henries'', not ``. . . a few H''.
\item Use a zero before decimal points: ``0.25'', not ``.25''. Use ``cm\textsuperscript{3}'', not ``cc''.)
\end{itemize}

\subsection{Equations}
Number equations consecutively. To make your 
equations more compact, you may use the solidus (~/~), the exp function, or 
appropriate exponents. Italicize Roman symbols for quantities and variables, 
but not Greek symbols. Use a long dash rather than a hyphen for a minus 
sign. Punctuate equations with commas or periods when they are part of a 
sentence, as in:
\begin{equation}
a+b=\gamma\label{eq}
\end{equation}

Be sure that the 
symbols in your equation have been defined before or immediately following 
the equation. Use ``\eqref{eq}'', not ``Eq.~\eqref{eq}'' or ``equation \eqref{eq}'', except at 
the beginning of a sentence: ``Equation \eqref{eq} is . . .''

\subsection{\LaTeX-Specific Advice}

Please use ``soft'' (e.g., \verb|\eqref{Eq}|) cross references instead
of ``hard'' references (e.g., \verb|(1)|). That will make it possible
to combine sections, add equations, or change the order of figures or
citations without having to go through the file line by line.

Please don't use the \verb|{eqnarray}| equation environment. Use
\verb|{align}| or \verb|{IEEEeqnarray}| instead. The \verb|{eqnarray}|
environment leaves unsightly spaces around relation symbols.

Please note that the \verb|{subequations}| environment in {\LaTeX}
will increment the main equation counter even when there are no
equation numbers displayed. If you forget that, you might write an
article in which the equation numbers skip from (17) to (20), causing
the copy editors to wonder if you've discovered a new method of
counting.

{\BibTeX} does not work by magic. It doesn't get the bibliographic
data from thin air but from .bib files. If you use {\BibTeX} to produce a
bibliography you must send the .bib files. 

{\LaTeX} can't read your mind. If you assign the same label to a
subsubsection and a table, you might find that Table I has been cross
referenced as Table IV-B3. 

{\LaTeX} does not have precognitive abilities. If you put a
\verb|\label| command before the command that updates the counter it's
supposed to be using, the label will pick up the last counter to be
cross referenced instead. In particular, a \verb|\label| command
should not go before the caption of a figure or a table.

Do not use \verb|\nonumber| inside the \verb|{array}| environment. It
will not stop equation numbers inside \verb|{array}| (there won't be
any anyway) and it might stop a wanted equation number in the
surrounding equation.

\subsection{Some Common Mistakes}\label{SCM}
\begin{itemize}
\item The word ``data'' is plural, not singular.
\item The subscript for the permeability of vacuum $\mu_{0}$, and other common scientific constants, is zero with subscript formatting, not a lowercase letter ``o''.
\item In American English, commas, semicolons, periods, question and exclamation marks are located within quotation marks only when a complete thought or name is cited, such as a title or full quotation. When quotation marks are used, instead of a bold or italic typeface, to highlight a word or phrase, punctuation should appear outside of the quotation marks. A parenthetical phrase or statement at the end of a sentence is punctuated outside of the closing parenthesis (like this). (A parenthetical sentence is punctuated within the parentheses.)
\item A graph within a graph is an ``inset'', not an ``insert''. The word alternatively is preferred to the word ``alternately'' (unless you really mean something that alternates).
\item Do not use the word ``essentially'' to mean ``approximately'' or ``effectively''.
\item In your paper title, if the words ``that uses'' can accurately replace the word ``using'', capitalize the ``u''; if not, keep using lower-cased.
\item Be aware of the different meanings of the homophones ``affect'' and ``effect'', ``complement'' and ``compliment'', ``discreet'' and ``discrete'', ``principal'' and ``principle''.
\item Do not confuse ``imply'' and ``infer''.
\item The prefix ``non'' is not a word; it should be joined to the word it modifies, usually without a hyphen.
\item There is no period after the ``et'' in the Latin abbreviation ``et al.''.
\item The abbreviation ``i.e.'' means ``that is'', and the abbreviation ``e.g.'' means ``for example''.
\end{itemize}
An excellent style manual for science writers is.

\subsection{Authors and Affiliations}
\textbf{The class file is designed for, but not limited to, six authors.} A 
minimum of one author is required for all conference articles. Author names 
should be listed starting from left to right and then moving down to the 
next line. This is the author sequence that will be used in future citations 
and by indexing services. Names should not be listed in columns nor group by 
affiliation. Please keep your affiliations as succinct as possible (for 
example, do not differentiate among departments of the same organization).

\subsection{Identify the Headings}
Headings, or heads, are organizational devices that guide the reader through 
your paper. There are two types: component heads and text heads.

Component heads identify the different components of your paper and are not 
topically subordinate to each other. Examples include Acknowledgments and 
References and, for these, the correct style to use is ``Heading 5''. Use 
``figure caption'' for your Figure captions, and ``table head'' for your 
table title. Run-in heads, such as ``Abstract'', will require you to apply a 
style (in this case, italic) in addition to the style provided by the drop 
down menu to differentiate the head from the text.

Text heads organize the topics on a relational, hierarchical basis. For 
example, the paper title is the primary text head because all subsequent 
material relates and elaborates on this one topic. If there are two or more 
sub-topics, the next level head (uppercase Roman numerals) should be used 
and, conversely, if there are not at least two sub-topics, then no subheads 
should be introduced.

\subsection{Figures and Tables}
\paragraph{Positioning Figures and Tables} Place figures and tables at the top and 
bottom of columns. Avoid placing them in the middle of columns. Large 
figures and tables may span across both columns. Figure captions should be 
below the figures; table heads should appear above the tables. Insert 
figures and tables after they are cited in the text. Use the abbreviation 
``Fig.~\ref{fig}'', even at the beginning of a sentence.

\begin{table}[htbp]
\caption{Table Type Styles}
\begin{center}
\begin{tabular}{|c|c|c|c|}
\hline
\textbf{Table}&\multicolumn{3}{|c|}{\textbf{Table Column Head}} \\
\cline{2-4} 
\textbf{Head} & \textbf{\textit{Table column subhead}}& \textbf{\textit{Subhead}}& \textbf{\textit{Subhead}} \\
\hline
copy& More table copy$^{\mathrm{a}}$& &  \\
\hline
\multicolumn{4}{l}{$^{\mathrm{a}}$Sample of a Table footnote.}
\end{tabular}
\label{tab1}
\end{center}
\end{table}

\begin{figure}[htbp]
\centerline{\includegraphics{fig1.png}}
\caption{Example of a figure caption.}
\label{fig}
\end{figure}

Figure Labels: Use 8 point Times New Roman for Figure labels. Use words 
rather than symbols or abbreviations when writing Figure axis labels to 
avoid confusing the reader. As an example, write the quantity 
``Magnetization'', or ``Magnetization, M'', not just ``M''. If including 
units in the label, present them within parentheses. Do not label axes only 
with units. In the example, write ``Magnetization (A/m)'' or ``Magnetization 
\{A[m(1)]\}'', not just ``A/m''. Do not label axes with a ratio of 
quantities and units. For example, write ``Temperature (K)'', not 
``Temperature/K''.

\section*{Acknowledgment}

The preferred spelling of the word ``acknowledgment'' in America is without 
an ``e'' after the ``g''. Avoid the stilted expression ``one of us (R. B. 
G.) thanks $\ldots$''. Instead, try ``R. B. G. thanks$\ldots$''. Put sponsor 
acknowledgments in the unnumbered footnote on the first page.

\section*{References}

Please number citations consecutively within brackets . The 
sentence punctuation follows the bracket . Refer simply to the reference 
number, as in ---do not use ``Ref. '' or ``reference '' except at 
the beginning of a sentence: ``Reference  was the first $\ldots$''

Number footnotes separately in superscripts. Place the actual footnote at 
the bottom of the column in which it was cited. Do not put footnotes in the 
abstract or reference list. Use letters for table footnotes.

Unless there are six authors or more give all authors' names; do not use 
``et al.''. Papers that have not been published, even if they have been 
submitted for publication, should be cited as ``unpublished'' . Papers 
that have been accepted for publication should be cited as ``in press'' . 
Capitalize only the first word in a paper title, except for proper nouns and 
element symbols.

For papers published in translation journals, please give the English 
citation first, followed by the original foreign-language citation .

\bibliographystyle{IEEEtran}
\bibliography{bibfile.bib}

\end{document}
